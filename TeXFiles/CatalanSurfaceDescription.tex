\documentclass[12pt,dvipdfmx]{article}
\usepackage{mathrsfs}
\usepackage{amsfonts}
\usepackage{amsmath}


\begin{document}

\begin{center}

{\bf \Large Catalan's minimal surface}

\end{center}


\section{Abstract}

Catalan's minimal surface is a minimal surface originally studied by Eug\`ene Charles Catalan in 1855.  
It has the special property of being the minimal surface that contains a cycloid as a geodesic. It is also swept out by a family of parabolae. (from Wikipeida)



\section{Definition}

It can be represented parametrically as

\begin{align*}
x(u,v)&	=	u-\sin u \cosh v	
\\
y(u,v)&	=	1-\cos u \cosh v	
\\
z(u,v)	&=	4 \sin \frac{u}{2} \sinh \frac{v}{2}
\end{align*}
for $u, v$ in $\mathbb{R}$.


\begin{thebibliography}{9}

\bibitem{Mathworld} MathWorld bt Wolfram, \verb|http://mathworld.wolfram.com/CatalansSurface.html|

\end{thebibliography}


\end{document}
