\documentclass[12pt,dvipdfmx]{article}
\usepackage{mathrsfs}
\usepackage{graphicx}
\usepackage{amsmath}


\begin{document}

\begin{center}

{\bf \Large Dini surface}

\end{center}


\section{Abstract}
In geometry, Dini's surface is a surface with constant negative curvature that can be created by twisting a pseudosphere. It is named after Ulisse Dini(1845 – 1918, Italia).

\section{Definition}
It is described by the following parametric equations:

\[
\begin{aligned}
x&=a\cos u\sin v \\
y&=a\sin u\sin v \\
z&=a\left(\cos v+\ln \tan {\frac {v}{2}}\right)+bu
\end{aligned}
\]
\begin{thebibliography}{9}

\bibitem{Mathworld} MathWorld by Wolfram, \\
 \verb|http://mathworld.wolfram.com/DinisSurface.html|


\end{thebibliography}


\end{document}
