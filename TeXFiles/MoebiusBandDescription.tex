\documentclass[12pt,dvipdfmx]{article}
\usepackage{mathrsfs}
\usepackage{graphicx}
\usepackage{amsmath}


\begin{document}

\begin{center}

{\bf \Large M\"{o}bius band}

\end{center}


\section{Abstract}
A M\"{o}bius band (also spelled Mobius or Moebius), is a surface with only one side (when embedded in three-dimensional Euclidean space) and only one boundary. The M\"{o}bius band has the mathematical property of being unorientable. It can be realized as a ruled surface. Its discovery is attributed to the German mathematicians Johann Benedict Listing and then independently August Ferdinand M\"{o}bius in 1858, though a structure similar to the M\"{o}bius band can be seen in Roman mosaics dated circa 200–250 AD.

An example of a M\"{o}bius strip can be created by taking a paper strip and giving it a half-twist, and then joining the ends of the strip to form a loop.

(Wikipedia)

\section{Definition}
One way to represent the M\"{o}bius strip as a subset of three-dimensional Euclidean space is using the parametrization:

\begin{align*}
 x(u,v)&=\left(1+{\frac {v}{2}}\cos {\frac {u}{2}}\right)\cos u \\
 y(u,v)&=\left(1+{\frac {v}{2}}\cos {\frac {u}{2}}\right)\sin u \\
 z(u,v)&=\frac {v}{2}\sin {\frac {u}{2}}
\end{align*}
where ${\displaystyle 0\leq u<2\pi }$ and ${\displaystyle -1\leq v\leq 1}$. 


\begin{thebibliography}{9}

\bibitem{Wikipedia}  \verb|https://en.wikipedia.org/wiki/Mobius_strip|


\end{thebibliography}


\end{document}