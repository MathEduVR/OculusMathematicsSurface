\documentclass[12pt,dvipdfmx]{article}
\usepackage{mathrsfs}
\usepackage{graphicx}
\usepackage{amsmath}


\begin{document}

\begin{center}

{\bf \Large ROMAN surface}

\end{center}


\section{Abstract}

The Roman surface, is a quartic nonorientable surface The Roman surface is one of the three possible surfaces obtained by sewing a Möbius strip to the edge of a disk. The other two are the Boy surface and cross-cap, all of which are homeomorphic to the real projective plane (Pinkall 1986). (fromMathWolrd)

The center point of the Roman surface is an ordinary triple point with $(\pm1,0,0)=(0, \pm 1,0)=(0,0, \pm 1)$, and the six endpoints of the three lines of self-intersection are singular pinch points, also known as pinch points. The Roman surface is essentially six cross-caps stuck together and contains a double infinity of conics. (fromMathWolrd)


The Roman surface is the quintic surface of revolution given by the equation
\[
 (x^2+y^2+z^2-k^2)^2=\{(z-k)^2-2x^2\}\{(z+k)^2-2y^2\}.  	
\]
that is closely related to the ding-dong surface. It is so named because the shape of the lower portion resembles that of a Hershey's Chocolate Kiss. (MathWorld)

\section{Definition}

It can be represented parametrically as

\begin{align*}
x(u,v)	&=	a\ sin(2u) \sin^2v	\\ 
y(u,v)	&=	a\sin u \sin(2v)	\\
z(u,v)	&=	a\cos u \sin(2v),
\end{align*}
for $u$ in $[0,2\pi)$ and $v$ in $[-\pi/2,\pi/2]$, where $a$ is a constant.


\begin{thebibliography}{9}

\bibitem{Mathworld} MathWorld bt Wolfram, \verb|http://mathworld.wolfram.com/RomanSurface.html|


\end{thebibliography}

\newpage

{\Huge $z=\dfrac{xy}{x^2+y^2}$}
\end{document}
