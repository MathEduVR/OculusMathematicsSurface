\documentclass[12pt,dvipdfmx]{article}
\usepackage{mathrsfs}
\usepackage{graphicx}
\usepackage{amsmath}


\begin{document}

\begin{center}

{\bf \Large KISS surface}

\end{center}


\section{Abstract}

The kiss surface is the quintic surface of revolution given by the equation
\[
 x^2+y^2=2(1-z)z^4 	
\]
that is closely related to the ding-dong surface. It is so named because the shape of the lower portion resembles that of a Hershey's Chocolate Kiss. (MathWorld)

\section{Definition}

It can be represented parametrically as

\begin{align*}
x(u,v)	&=	av^2\sqrt{(1-v)/2}\cos u	\\
y(u,v)	&=	av^2\sqrt{(1-v)/2}\sin u	\\
z(u,v)	&=	av,
\end{align*}
where $a$ is a constant.


\begin{thebibliography}{9}

\bibitem{Mathworld} MathWorld bt Wolfram, \verb|http://mathworld.wolfram.com/KissSurface.html|

\bibitem{MathEncyc} Concise Encyclopedia of Mathematics by Eric W. Weisstein, \verb|https://archive.lib.msu.edu/crcmath/math/math/k/k079.htm|

\end{thebibliography}


\end{document}
