\documentclass[12pt,dvipdfmx]{article}
\usepackage{mathrsfs}
\usepackage{graphicx}
\usepackage{amsmath}


\begin{document}

\begin{center}

{\bf \Large BOY surface}

\end{center}


\section{Abstract}

Boy's surface is an immersion of the real projective plane in 3-dimensional space found by Werner Boy in 1901. He discovered it on assignment from David Hilbert to prove that the projective plane could not be immersed in 3-space.(from Wikipedia)

The Boy surface is a nonorientable surface that is one possible parametrization of the surface obtained by sewing a Moebius strip to the edge of a disk. Two other topologically equivalent parametrizations are the cross-cap and Roman surface. The Boy surface is a model of the projective plane without singularities and is a sextic surface. (from MathWorld)

\section{Definition}

It can be represented parametrically as

\begin{align*}
x	&=	\dfrac{\sqrt{2} \cos^2v \cos(2u)+ \cos u \sin(2v)}{2-\sqrt{2} \sin(3u) \sin(2v)}	
\\
y	&=	\dfrac{sqrt{2}cos^2v \sin(2u)- \sin u \sin(2v)}{2-\sqrt{2} \sin(3u) \sin(2v)}	
\\
z	&=	\dfrac{3 \cos^2v}{2-\sqrt{2} \sin(3u) \sin(2v)}
\end{align*}
for $u$ in $\left[-\frac{\pi}{2},\frac{\pi}{2} \right]$ and $v$ in $[0, \pi]$.

There exists a homotopy (smooth deformation) between the Roman surface and Boy surface as a parameter $\alpha$ varies from $0$ to $1$, where $\alpha=0$ corresponds to the Roman surface and $\alpha=1$ to the Boy surface.
\begin{thebibliography}{9}

\bibitem{Mathworld} MathWorld bt Wolfram, \verb|http://mathworld.wolfram.com/BoySurface.html|


\end{thebibliography}


\end{document}
